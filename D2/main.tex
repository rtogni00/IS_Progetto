% !TeX document-id = {87432cc9-c386-48dc-9558-b3c42a74a150}
% !TEX encoding = UTF-8
% !TEX TS-program = pdflatex
% !TEX spellcheck = it-IT
%% !TEX root = 
% !BIB TS-program = biber
% Comandi speciali (o righe magiche) trattati come commenti da Latex ma non da 
%editor come Texstudio e Texshop, che si impostano di conseguenza. In ordine:
% dichiara la codifica dei caratteri con cui impostare l'editor (bisogna 
%comunque caricare inputenc con la stessa codifica)
% dichiara il motore di composizione (pdflatex o lualatex)
% attiva il controllo ortografico della lingua del documento
% dichiara la condizione di file principale e di file secondario nei documenti 
%suddivisi in più file
% imposta biber come motore bibliografico

\documentclass[9pt]{extarticle}


% ------------------------------------------------------------------------------
% Packages
% ------------------------------------------------------------------------------
\usepackage[utf8]{inputenc}
\usepackage{graphicx}
\usepackage{amsmath}
\usepackage{amssymb}
\usepackage{amsthm}
\usepackage{hyperref}
\usepackage{titlesec}
\usepackage{parskip}
\usepackage{float}
\usepackage{booktabs}
\usepackage{caption}
\usepackage{blindtext}
\usepackage[table,xcdraw]{xcolor}
\usepackage{booktabs} % For better table rules
\usepackage{array} % For better column definitions
\usepackage{enumitem}
%\usepackage{svg}
% Define a new counter for functional requirements
\newcounter{rf}
\newcommand{\FR}{\refstepcounter{rf}\textbf{FR\therf: }}
% Define a new counter for non-functional requirements
\newcounter{nonfunc}
\newcommand{\NONFUNC}{\refstepcounter{nonfunc}\textbf{NFR\thenonfunc: }}

% ------------------------------------------------------------------------------
% Page setup
% ------------------------------------------------------------------------------
\usepackage{geometry} 
\geometry{a4paper, margin=1in, twoside}
\usepackage{setspace}
\setstretch{1.2}  % Adjust the number as needed
%\doublespacing


% ------------------------------------------------------------------------------
% Referencing
% ------------------------------------------------------------------------------
\usepackage[
	backend=biber,
	citestyle=authoryear,
	bibstyle=authoryear,
	maxcitenames=2,
	maxbibnames=99]{biblatex}
%\addbibresource{references.bib}
%\DeclareNameAlias{sortname}{family-given}
%\setlength\bibitemsep{1em}

% ------------------------------------------------------------------------------
% Code listings
% ------------------------------------------------------------------------------
\usepackage{listings}
\usepackage{xcolor}
\lstdefinestyle{matlabcode}{
	backgroundcolor=\color{gray!10},   
	commentstyle=\color{green!50!black},
	keywordstyle=\color{blue},
	stringstyle=\color{magenta},
	basicstyle=\linespread{1}\footnotesize\ttfamily,
	numberstyle=\tiny,
	breakatwhitespace=false,         
	breaklines=true,                 
	captionpos=t,   
	frame=single,
	keepspaces=true,         
	language=matlab,        
	numbers=none,             
	numbersep=5pt,                  
	showspaces=false,                
	showstringspaces=false,
	showtabs=false,                  
	tabsize=2,
	aboveskip=1em,
	belowskip=1em,
	belowcaptionskip=12pt
}

\lstdefinelanguage{yaml}{
	keywords={true,false,null,y,n},
	keywordstyle=\color{blue},
	basicstyle=\linespread{1}\footnotesize\ttfamily,
	sensitive=false,
	comment=[l]{\#},
	morecomment=[s]{/*}{*/},
	commentstyle=\color{green!50!black},
	stringstyle=\color{magenta},
	showstringspaces=false,
	breaklines=true
}

\lstdefinestyle{yamlcode}{
	backgroundcolor=\color{gray!10},   
	commentstyle=\color{green!50!black},
	keywordstyle=\color{blue},
	stringstyle=\color{magenta},
	basicstyle=\linespread{1}\footnotesize\ttfamily,
	numberstyle=\tiny,
	breakatwhitespace=false,         
	breaklines=true,                 
	captionpos=b,  % 't' for top 
	frame=single,
	keepspaces=true,         
	language=YAML,        
	numbers=none,             
	numbersep=5pt,                  
	showspaces=false,                
	showstringspaces=false,
	showtabs=false,                  
	tabsize=2,
	aboveskip=1em,
	belowskip=1em,
	belowcaptionskip=12pt
}

\lstdefinestyle{treestyle}{
	backgroundcolor=\color{gray!10},   
	basicstyle=\ttfamily\footnotesize,  % Smaller font, monospaced
	frame=single,                       % Add a border around the listing
	breaklines=true,                    
	showstringspaces=false,             
	tabsize=4,                          
	captionpos=b,                       
}

\lstdefinelanguage{JavaScript}{
	morekeywords={typeof, new, true, false, catch, function, return, null, catch, switch, var, let, const, if, in, while, do, else, case, break, for, try, throw, class, extends, super, import, export, default, async, await},
	morecomment=[l]{//},       % Line comments
	morecomment=[s]{/*}{*/},   % Block comments
	morestring=[b]",           % Double quote strings
	morestring=[b]',           % Single quote strings
	morestring=[b]`            % Template literals (backticks)
}

\lstdefinestyle{jscode}{
	backgroundcolor=\color{gray!10},   
	commentstyle=\color{green!50!black},
	keywordstyle=\color{blue},
	stringstyle=\color{magenta},
	numberstyle=\tiny,
	basicstyle=\linespread{1}\footnotesize\ttfamily,
	breakatwhitespace=false,         
	breaklines=true,                 
	captionpos=b,  % 't' for top
	frame=single,
	keepspaces=true,         
	language=JavaScript,        
	numbers=none,             
	numbersep=5pt,                  
	showspaces=false,                
	showstringspaces=false,
	showtabs=false,                  
	tabsize=2,
	aboveskip=1em,
	belowskip=1em,
	belowcaptionskip=12pt
}



% ------------------------------------------------------------------------------
% Headers and footers
% ------------------------------------------------------------------------------	
\usepackage{fancyhdr}
\pagestyle{fancy}   
\setlength{\headheight}{14.5pt}
\renewcommand{\headrulewidth}{0.3pt}
%\renewcommand{\chaptermark}[1]{\markboth{\chaptername\ \thechapter.\ #1}{}}
\renewcommand{\sectionmark}[1]{\markright{\thesection.\ #1}}   
\fancyhead[LE,RO]{\rightmark}
\fancyhead[LO,RE]{\leftmark}

% ------------------------------------------------------------------------------
% Title page
% ------------------------------------------------------------------------------

\newcommand{\name}{Togni Roberto}
\newcommand{\projecttitle}{EvenTrento}
\newcommand{\course}{Ingegneria del Software}
\newcommand{\customtitle}{
	\vspace*{1cm}
	\begin{center}
		\includegraphics[width = 7cm]{images/logo_rosso} \\
		\vspace{1cm}
	\end{center}
	\LARGE{Progetto:}
	\begin{center}
%		\vspace{2cm}
		\textbf{\Huge{\projecttitle}} \\
	\end{center}
	\LARGE{Titolo del documento:}
	\begin{center}
%		\vspace{1cm}
		\textbf{\Huge Implementazione} \\
	\end{center}
	\LARGE{Autore:}
	\begin{center}
		%		\vspace{1cm}
		\textbf{\name} \\
	\end{center}
	\vspace{1cm}
	Document Info:
	
	\begingroup
	\setlength{\tabcolsep}{10pt} % Default value: 6pt
	\renewcommand{\arraystretch}{1.5}
	
	\begin{table}[!htb]
		\begin{tabular}{lllll}
			\cline{1-4}
			\cellcolor[HTML]{13315C}{\color[HTML]{FFFFFF} Doc. Name}                         & D3-EvenTrentoImplementazione & \multicolumn{1}{l|}{\cellcolor[HTML]{13315C}{\color[HTML]{FFFFFF} Doc. Number}} & \multicolumn{1}{l|}{D3 V0.1} &  \\ \cline{1-4}
			\multicolumn{1}{|l|}{\cellcolor[HTML]{13315C}{\color[HTML]{FFFFFF} Description}} & \multicolumn{3}{l|}{Documento di descizione dell'implementazione}                                                    &  \\ \cline{1-4}
			&                                  &                                                                                 &                              &  \\
			&                                  &                                                                                 &                              &  \\
		\end{tabular}
	\end{table}
	
	\endgroup
	
	\begin{center}
		\vfill 
		\textbf{\Large Dipartimento di Ingegneria e Scienza dell'Informazione}
		\vspace{1cm}
	\end{center}
	\newpage
	\pagenumbering{roman}
	\setcounter{page}{0}
}

% ------------------------------------------------------------------------------
% Theorem enivornments
% ------------------------------------------------------------------------------
%\newtheorem{theorem}{Theorem}[chapter]
%\newtheorem{definition}{Definition}[chapter]
%\newtheorem{corollary}{Corollary}[theorem]
%\newtheorem{lemma}[theorem]{Lemma}

\newtheoremstyle{break}%
    {}{}%
    {}{}%
    {\bfseries}{}% % Note that final punctuation is omitted.
    {\newline}{}
\theoremstyle{break}
%\newtheorem{example}{Example}[chapter]



% ------------------------------------------------------------------------------
% Document
% ------------------------------------------------------------------------------


\begin{document}
\customtitle



% TODO add chapters
\tableofcontents
\newpage

\section{Scopo del documento}


Il seguente documento riporta la specifica dei requisiti funzionali del sistema tramite un linguaggio semi-formale. Si tratta dunque di un approfondimento (nonché di una formalizzazione) di quanto riportato in linguaggio naturale all'interno del D1. Il linguaggio utilizzato per la formalizzazione dei requisiti è UML (Unified Modeling Language), declinato in Use Case Diagrams (UCDs), Component Diagrams, Sequence Diagrams, e Class Diagrams.

\newpage

\section{Requisiti Funzionali}

Di seguito sono riportati i functional requirements (FR) del sistema sia in linguaggio naturale che tramite Use Case Diagrams (UCDs). La notazione è coerente con quella utilizzata all'interno del documento D1.



\subsection*{FR1 e FR2: Login e Registrazione} 

\begin{figure}[!htb]
	\centering
	\includegraphics[width=.7\linewidth]{./images/FR1-2.pdf}
	\caption{UCD relativo a FR1 e FR2.}
	\label{fig:UCD_FR1-2}
\end{figure}

\subsubsection*{Use Case FR1: Login}


\begin{itemize}
	\item L'utente visualizza una pagina con un form
	\item Se l'utente inserisce le credenziali (vedi  \hyperref[Estensioni-FR2]{Estensioni}) corrette (vedi \hyperref[Eccezioni-FR2]{Eccezioni}) e preme sul pulsante "Login", si apre la schermata principale dell'applicazione.
	\item Qualora l'utente selezioni il logo di Google, il processo di autenticazione verrà gestito da Google SSO
\end{itemize}

\subsubsection*{Use Case FR2: Registrazione}

\begin{itemize}
	\item L'utente visualizza una pagina dalla quale inserire nome, cognome, indirizzo mail (vedi \hyperref[Eccezioni-FR2]{Eccezioni})
	\item Il sistema invia una mail all'indirizzo fornito dall'utente contenente un link e una password temporanea
	\item L'utente deve confermare la registrazione tramite il link di cui sopra (vedi \hyperref[Eccezioni-FR2]{Eccezioni}). Può quindi accedere al sistema tramite la password temporanea
\end{itemize}


\subsubsection*{Eccezioni}\label{Eccezioni-FR2}

\begin{itemize}
	\item Qualora le credenziali inserite non siano corrette, l'applicazione restituisce un messaggio di errore
	\item Qualora l'utente non apra il link contenuto nella mail automatica, la registrazione non viene finalizzata
\end{itemize}

\subsubsection*{Estensioni}\label{Estensioni-FR2}

\begin{itemize}
	\item La password contenuta nella mail generata automaticamente dal sistema a seguito della registrazione dev'essere modificata a seguito del primo login
\end{itemize}

\subsection*{FR4: Visualizzazione Eventi}

\begin{figure}[!htb]
	\centering
	\includegraphics[width=.7\linewidth]{./images/FR4.pdf}
	\caption{UCD relativo al FR4.}
	\label{fig:UCD_FR4}
\end{figure}

\subsubsection*{Use Case FR34: Visualizzazione Eventi}

\begin{itemize}
	\item La ricerca di eventi può essere effettuata da qualsiasi utente, anche non loggato
	\item L'utente può scegliere se esplorare la mappa integrata nella schermata iniziale, o se visualizzare gli eventi in forma di lista
	\item Qualora l'utente decida di visualizzare la lista degli eventi, viene messo a disposizione un servizio di filtering basato su 3 possibili criteri: data, luogo e nome dell'evento
\end{itemize}

\subsection*{FR8 e FR10: Creazione ed Aggiornamento Eventi}

\begin{figure}[!htb]
	\centering
	\includegraphics[width=\linewidth]{./images/FR8-10.pdf}
	\caption{UCD relativo a FR8 e FR10.}
	\label{fig:UCD_FR8-10}
\end{figure}

\subsubsection*{Use Case FR8: Creazione Eventi}

\begin{itemize}
	\item Qualora l'utente appartenga alla categoria "Owner" o a quella "Organizer" e sia loggato (vedi \hyperref[Eccezioni-FR8-10]{Eccezioni}), dalla pagina del profilo personale è visibile un pulsante "New event"
	\item La pressione del pulsante rimanda alla pagina di creazione eventi. Per la creazione di un nuovo evento è necessario l'inserimento di un nome, di una data, di una location e di una descrizione (vedi \hyperref[Eccezioni-FR8-10]{Eccezioni})
	\item L'aggiunta di immagini è facoltativa, così come lo è il prezzo 
\end{itemize}

\subsubsection*{Use case FR10: Modifica Eventi}
\begin{itemize}
	\item Qualora un utente appartenente alle categorie "Owner" o "Organizer" sia loggato e abbia precedentemente creato un evento (vedi \hyperref[Eccezioni-FR8-10]{Eccezioni}), dalla propria pagina personale può raggiungere l'evento in questione tramite il pulsante "My events"
	\item La pressione del pulsante rimanda ad una lista degli eventi creati. Selezionandone uno è possibile modificare uno qualsiasi dei vari campi entro una settimana dall'evento (vedi \hyperref[Eccezioni-FR8-10]{Eccezioni} e \hyperref[Estensioni-FR8-10]{Estensioni})
\end{itemize}

\subsubsection*{Eccezioni}\label{Eccezioni-FR8-10}

\begin{itemize}
	\item Qualora l'utente non sia loggato, non esiste alcun profilo personale
	\item Qualora l'utente appartenga alla categoria "User", la pagina del profilo personale è priva del pulsante per la creazione di eventi
	\item Se durante la creazione di un evento non viene compilato uno dei campi obbligatori, il pulsante "Create Event" rimane inattivo
	\item Qualora l'utente non abbia mai creato eventi, il pulsante "My Events" non è attivo
	\item Qualora manchi meno di una settimana all'evento in questione, i campi non risultano modificabili
\end{itemize}

\subsubsection*{Estensioni}\label{Estensioni-FR8-10}

\begin{itemize}
	\item Gli eventi passati rimangono visualizzabili tramite il pulsante "My events", tuttavia non risultano più editabili
\end{itemize}

\newpage

\subsection*{FR12-13: Aggiunta e Modifica Luogo}


\begin{figure}[!htb]
	\centering
	\includegraphics[width=\linewidth]{./images/FR12.pdf}
	\caption{UCD relativo al FR12.}
	\label{fig:UCD_FR12}
\end{figure}

\subsubsection*{Use Case FR12: Aggiunta Luogo}
\begin{itemize}
	\item Qualora l'utente sia loggato con un profilo di tipo "Owner" (vedi \hyperref[Eccezioni-FR12-13]{Eccezioni}), dal proprio profilo personale ha la possibilità di aggiungere un luogo tramite il pulsante "New Place"
	\item La pressione del pulsante rimanda alla pagina di aggiunta luoghi. Per l'aggiunta di un nuovo luogo occorre inserire una location, un nome, una descrizione, ed un prezzo (vedi \hyperref[Eccezioni-FR12-13]{Eccezioni}). Opzionalmente è possibile inserire delle foto
\end{itemize}

\subsubsection*{Use Case FR13: Modifica Luogo}

\begin{itemize}
	\item Qualora l'utente sia loggato con un profilo di tipo "Owner" e abbia precedentemente aggiunto una location, dalla propria pagina personale può raggiungere l'evento in questione tramite il pulsante "My places"
	\item La pressione del pulsante rimanda ad una lista degli spazi aggiunti. Selezionandone uno è possibile modificare uno qualsiasi dei campi e/o eliminare lo spazio se e solo se lo spazio non è attualmente collegato ad un evento (vedi \hyperref[Eccezioni-FR12-13]{Eccezioni})
\end{itemize}

\subsubsection*{Eccezioni}\label{Eccezioni-FR12-13}
\begin{itemize}
	\item Qualora l'utente non sia loggato, non esiste alcun profilo personale
	\item Qualora l'utente non appartenga alla categoria "Owner", la pagina del profilo personale è priva del pulsante per l'aggiunta di spazi
	\item Se durante l'aggiunta di uno spazio non viene compilato uno dei campi obbligatori, il pulsante "Create Place" rimane inattivo
	\item Qualora lo spazio sia collegato ad un evento, le modifiche sono disattivate a partire da 7 giorni prima della data prevista per l'evento
\end{itemize}

\section{Analisi dei Componenti}

\subsection{Definizione dei Componenti}


\subsection{Diagramma dei Componenti}

\section{Diagramma delle Classi}




	
\end{document}