% !TeX document-id = {87432cc9-c386-48dc-9558-b3c42a74a150}
% !TEX encoding = UTF-8
% !TEX TS-program = pdflatex
% !TEX spellcheck = it-IT
%% !TEX root = 
% !BIB TS-program = biber
% Comandi speciali (o righe magiche) trattati come commenti da Latex ma non da 
%editor come Texstudio e Texshop, che si impostano di conseguenza. In ordine:
% dichiara la codifica dei caratteri con cui impostare l'editor (bisogna 
%comunque caricare inputenc con la stessa codifica)
% dichiara il motore di composizione (pdflatex o lualatex)
% attiva il controllo ortografico della lingua del documento
% dichiara la condizione di file principale e di file secondario nei documenti 
%suddivisi in più file
% imposta biber come motore bibliografico

\documentclass[9pt]{extarticle}


% ------------------------------------------------------------------------------
% Packages
% ------------------------------------------------------------------------------
\usepackage[utf8]{inputenc}
\usepackage{graphicx}
\usepackage{amsmath}
\usepackage{amssymb}
\usepackage{amsthm}
\usepackage{hyperref}
\usepackage{titlesec}
\usepackage{parskip}
\usepackage{float}
\usepackage{booktabs}
\usepackage{caption}
\usepackage{blindtext}
\usepackage[table,xcdraw]{xcolor}
\usepackage{booktabs} % For better table rules
\usepackage{array} % For better column definitions
\usepackage{enumitem}
%\usepackage{svg}
% Define a new counter for functional requirements
\newcounter{rf}
\newcommand{\FR}{\refstepcounter{rf}\textbf{FR\therf: }}
% Define a new counter for non-functional requirements
\newcounter{nonfunc}
\newcommand{\NONFUNC}{\refstepcounter{nonfunc}\textbf{NFR\thenonfunc: }}

% ------------------------------------------------------------------------------
% Page setup
% ------------------------------------------------------------------------------
\usepackage{geometry} 
\geometry{a4paper, margin=1in, twoside}
\usepackage{setspace}
\setstretch{1.2}  % Adjust the number as needed
%\doublespacing


% ------------------------------------------------------------------------------
% Referencing
% ------------------------------------------------------------------------------
\usepackage[
	backend=biber,
	citestyle=authoryear,
	bibstyle=authoryear,
	maxcitenames=2,
	maxbibnames=99]{biblatex}
%\addbibresource{references.bib}
%\DeclareNameAlias{sortname}{family-given}
%\setlength\bibitemsep{1em}

% ------------------------------------------------------------------------------
% Code listings
% ------------------------------------------------------------------------------
\usepackage{listings}
\usepackage{xcolor}
\lstdefinestyle{matlabcode}{
	backgroundcolor=\color{gray!10},   
	commentstyle=\color{green!50!black},
	keywordstyle=\color{blue},
	stringstyle=\color{magenta},
	basicstyle=\linespread{1}\footnotesize\ttfamily,
	numberstyle=\tiny,
	breakatwhitespace=false,         
	breaklines=true,                 
	captionpos=t,   
	frame=single,
	keepspaces=true,         
	language=matlab,        
	numbers=none,             
	numbersep=5pt,                  
	showspaces=false,                
	showstringspaces=false,
	showtabs=false,                  
	tabsize=2,
	aboveskip=1em,
	belowskip=1em,
	belowcaptionskip=12pt
}

\lstdefinelanguage{yaml}{
	keywords={true,false,null,y,n},
	keywordstyle=\color{blue},
	basicstyle=\linespread{1}\footnotesize\ttfamily,
	sensitive=false,
	comment=[l]{\#},
	morecomment=[s]{/*}{*/},
	commentstyle=\color{green!50!black},
	stringstyle=\color{magenta},
	showstringspaces=false,
	breaklines=true
}

\lstdefinestyle{yamlcode}{
	backgroundcolor=\color{gray!10},   
	commentstyle=\color{green!50!black},
	keywordstyle=\color{blue},
	stringstyle=\color{magenta},
	basicstyle=\linespread{1}\footnotesize\ttfamily,
	numberstyle=\tiny,
	breakatwhitespace=false,         
	breaklines=true,                 
	captionpos=b,  % 't' for top 
	frame=single,
	keepspaces=true,         
	language=YAML,        
	numbers=none,             
	numbersep=5pt,                  
	showspaces=false,                
	showstringspaces=false,
	showtabs=false,                  
	tabsize=2,
	aboveskip=1em,
	belowskip=1em,
	belowcaptionskip=12pt
}

\lstdefinestyle{treestyle}{
	backgroundcolor=\color{gray!10},   
	basicstyle=\ttfamily\footnotesize,  % Smaller font, monospaced
	frame=single,                       % Add a border around the listing
	breaklines=true,                    
	showstringspaces=false,             
	tabsize=4,                          
	captionpos=b,                       
}

\lstdefinelanguage{JavaScript}{
	morekeywords={typeof, new, true, false, catch, function, return, null, catch, switch, var, let, const, if, in, while, do, else, case, break, for, try, throw, class, extends, super, import, export, default, async, await},
	morecomment=[l]{//},       % Line comments
	morecomment=[s]{/*}{*/},   % Block comments
	morestring=[b]",           % Double quote strings
	morestring=[b]',           % Single quote strings
	morestring=[b]`            % Template literals (backticks)
}

\lstdefinestyle{jscode}{
	backgroundcolor=\color{gray!10},   
	commentstyle=\color{green!50!black},
	keywordstyle=\color{blue},
	stringstyle=\color{magenta},
	numberstyle=\tiny,
	basicstyle=\linespread{1}\footnotesize\ttfamily,
	breakatwhitespace=false,         
	breaklines=true,                 
	captionpos=b,  % 't' for top
	frame=single,
	keepspaces=true,         
	language=JavaScript,        
	numbers=none,             
	numbersep=5pt,                  
	showspaces=false,                
	showstringspaces=false,
	showtabs=false,                  
	tabsize=2,
	aboveskip=1em,
	belowskip=1em,
	belowcaptionskip=12pt
}



% ------------------------------------------------------------------------------
% Headers and footers
% ------------------------------------------------------------------------------	
\usepackage{fancyhdr}
\pagestyle{fancy}   
\setlength{\headheight}{14.5pt}
\renewcommand{\headrulewidth}{0.3pt}
%\renewcommand{\chaptermark}[1]{\markboth{\chaptername\ \thechapter.\ #1}{}}
\renewcommand{\sectionmark}[1]{\markright{\thesection.\ #1}}   
\fancyhead[LE,RO]{\rightmark}
\fancyhead[LO,RE]{\leftmark}

% ------------------------------------------------------------------------------
% Title page
% ------------------------------------------------------------------------------

\newcommand{\name}{Togni Roberto}
\newcommand{\projecttitle}{EvenTrento}
\newcommand{\course}{Ingegneria del Software}
\newcommand{\customtitle}{
	\vspace*{1cm}
	\begin{center}
		\includegraphics[width = 7cm]{images/logo_rosso} \\
		\vspace{1cm}
	\end{center}
	\LARGE{Progetto:}
	\begin{center}
%		\vspace{2cm}
		\textbf{\Huge{\projecttitle}} \\
	\end{center}
	\LARGE{Titolo del documento:}
	\begin{center}
%		\vspace{1cm}
		\textbf{\Huge Implementazione} \\
	\end{center}
	\LARGE{Autore:}
	\begin{center}
		%		\vspace{1cm}
		\textbf{\name} \\
	\end{center}
	\vspace{1cm}
	Document Info:
	
	\begingroup
	\setlength{\tabcolsep}{10pt} % Default value: 6pt
	\renewcommand{\arraystretch}{1.5}
	
	\begin{table}[!htb]
		\begin{tabular}{lllll}
			\cline{1-4}
			\cellcolor[HTML]{13315C}{\color[HTML]{FFFFFF} Doc. Name}                         & D3-EvenTrentoImplementazione & \multicolumn{1}{l|}{\cellcolor[HTML]{13315C}{\color[HTML]{FFFFFF} Doc. Number}} & \multicolumn{1}{l|}{D3 V0.1} &  \\ \cline{1-4}
			\multicolumn{1}{|l|}{\cellcolor[HTML]{13315C}{\color[HTML]{FFFFFF} Description}} & \multicolumn{3}{l|}{Documento di descizione dell'implementazione}                                                    &  \\ \cline{1-4}
			&                                  &                                                                                 &                              &  \\
			&                                  &                                                                                 &                              &  \\
		\end{tabular}
	\end{table}
	
	\endgroup
	
	\begin{center}
		\vfill 
		\textbf{\Large Dipartimento di Ingegneria e Scienza dell'Informazione}
		\vspace{1cm}
	\end{center}
	\newpage
	\pagenumbering{roman}
	\setcounter{page}{0}
}

% ------------------------------------------------------------------------------
% Theorem enivornments
% ------------------------------------------------------------------------------
%\newtheorem{theorem}{Theorem}[chapter]
%\newtheorem{definition}{Definition}[chapter]
%\newtheorem{corollary}{Corollary}[theorem]
%\newtheorem{lemma}[theorem]{Lemma}

\newtheoremstyle{break}%
    {}{}%
    {}{}%
    {\bfseries}{}% % Note that final punctuation is omitted.
    {\newline}{}
\theoremstyle{break}
%\newtheorem{example}{Example}[chapter]



% ------------------------------------------------------------------------------
% Document
% ------------------------------------------------------------------------------


\begin{document}
\customtitle



% TODO add chapters
\tableofcontents
\newpage

\section{User Stories}

Le seguenti tabelle raccolgono una sere di User Stories divise in diverse epiche. Ad ogn user story è associata una priorità (ad un numero più piccolo corrisponde una priorità maggiore) rappresentante l'importanza della funzionalità in questione per l'operatività del sistema.

	
\begin{table}[!htb]
	\centering
	\begin{tabular}{clp{7cm}l} % Use 'p{7cm}' for the User Story column
		\toprule
		\multicolumn{4}{c}{\textbf{Profilo}}\\ \midrule
		Id & Nome & User Story & Priorità \\ \midrule
		A1  & Registrazione  & Come utente, voglio registrarmi per utilizzare il sistema & 1 \\
		A2  & Login & In qualità di utente, devo essere in grado di effettuare il login & 1 \\
		A3  & Visualizzazione profilo & Come utente, devo essere in grado di visualizzare lo storico eventi e le mie informazioni personali &  \\
		\bottomrule
	\end{tabular}
	\caption{User stories relative all'epica "profilo".}
	\label{tab:profili}
\end{table}


\begin{table}[!htb]
	\centering
	\begin{tabular}{clp{7cm}l} % Use 'p{7cm}' for the User Story column
		\toprule
		\multicolumn{4}{c}{\textbf{Eventi}}\\ \midrule
		Id & Nome & User Story & Priorità \\ \midrule
		B1  & Lista eventi            & In qualità di utente, devo essere in grado di scorrere una lista di eventi                                               &  \\
		B2  & Filtri eventi           & In qualità di utente, devo essere in grado di filtrare gli eventi in base alla data                                      &  \\
		B3  & Creazione evento        & In qualità di organizzatore, devo essere in grado di creare un nuovo evento in un determinato luogo                      &  2 \\
		B4  & Condivisione evento     & In qualità di utente, devo essere in grado di condividere un evento tramite link                                         &  \\
		B5  & Pagamento               & In qualità di utente, devo essere in grado di effettuare il pagamento per un evento a cui voglio iscrivermi              & \\
		B6  & Aggiornamento evento    & In qualità di organizzatore, devo essere in grado di aggiornare la descrizione di un evento                              & \\
		B7  & Statistiche evento      & In qualità di organizzatore, devo essere in grado di visualizzare gli iscritti all'evento                                & \\
		B8  & Salvataggio evento      & In qualità di utente, devo essere in grado di salvare un evento in modo da poterlo visualizzare nella mia area personale & \\
		\bottomrule
	\end{tabular}
	\caption{User stories relative all'epica "eventi".}
	\label{tab:eventi}
\end{table}

\begin{table}[!htb]
	\centering
	\begin{tabular}{clp{7cm}l} % Use 'p{7cm}' for the User Story column
		\toprule
		\multicolumn{4}{c}{\textbf{Spazi}}\\ \midrule
		Id & Nome & User Story & Priorità \\ \midrule
		C1  & Aggiunta spazio  & In qualità di owner, devo essere in grado di pubblicare uno spazio disponibile & 2\\
		C2 & Rimozione spazio & In qualità di owner, devo essere in grado di rimuovere uno spazio esistente & 2\\
		C3 & Visualizzazione spazi & In qualità di organizzatore, devo essere in grado di visualizzare gli spazi disponibili & \\
		\bottomrule
	\end{tabular}
	\caption{User stories relative all'epica "spazi".}
	\label{tab:spazi}
\end{table}



\begin{table}[!htb]
	\centering
	\begin{tabular}{clp{7cm}l} % Use 'p{7cm}' for the User Story column
		\toprule
		\multicolumn{4}{c}{\textbf{Mappa}}\\ \midrule
		Id & Nome & User Story & Priorità \\ \midrule
		C1  & Esplorazione mappa      & In qualità di utente, devo essere in grado di muovermi nella mappa per visualizzare gli eventi                           &  \\
		C2  & Aggiunta luogo          & In qualità di owner, devo essere in grado di aggiungere un nuovo luogo alla mappa                                        & \\
		\bottomrule
	\end{tabular}
	\caption{User stories relative all'epica "mappa".}
	\label{tab:mappa}
\end{table}

\newpage
\section{User Flow}

Di seguito sono riportati i diagrammi di flusso rappresentanti gli user flow corrispondenti ad una serie di user stories. La notazione utilizzata nelle descrizioni si riferisce a quella riportata nelle Tabelle \ref{tab:profili}, \ref{tab:eventi}, \ref{tab:spazi} e \ref{tab:mappa}.

\begin{figure}[!htb]
	\centering
	\includegraphics[width=0.8\linewidth]{./images/A1-A2.pdf}
	\caption{User Flowchart relativo alle user stories A1 e A2.}
	\label{fig:A1-A2}
\end{figure}

\begin{figure}[!htb]
	\centering
	\includegraphics[width=\linewidth]{./images/EventSelection.pdf}
	\caption{User Flowchart relativo alle user stories B1 e C1.}
	\label{fig:B1-C1}
\end{figure}

\newpage

\begin{figure}[!htb]
	\centering
	\includegraphics[width=0.8\linewidth]{./images/B5.pdf}
	\caption{User Flowchart relativo alla user story B5.}
	\label{fig:B5}
\end{figure}

	
\end{document}